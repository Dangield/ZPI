\documentclass[a4paper,titlepage,12pt]{article}
\usepackage[left=2.5cm,right=2.5cm,top=2.5cm,bottom=2.5cm]{geometry}
\usepackage[OT1]{fontenc}
\usepackage{polski}
\usepackage{amsmath}
\usepackage{amsfonts}
\usepackage{amssymb}
\usepackage{graphicx}
\usepackage{url}
\usepackage{tikz}
\usetikzlibrary{arrows,calc,decorations.markings,math,arrows.meta}
\usepackage{rotating}
\usepackage[percent]{overpic}
\usepackage[cp1250]{inputenc}
\usepackage{xcolor}
\usepackage{pgfplots}
\usetikzlibrary{pgfplots.groupplots}
\usepackage{listings}
\usepackage{matlab-prettifier}
\usepackage{siunitx}
\usepackage[section]{placeins}

\usepackage{fancyhdr}
\usepackage{indentfirst}
\usepackage{makecell}
\newcommand{\tabitem}{~~\llap{\textbullet}~~}
\usepackage{colortbl}

\definecolor{szary}{rgb}{0.95,0.95,0.95}
\sisetup{detect-weight,exponent-product=\cdot,output-decimal-marker={,},per-mode=symbol,binary-units=true,range-phrase={-},range-units=single}
\SendSettingsToPgf

%konfiguracje pakietu listings
\lstset{
	backgroundcolor=\color{szary},
	frame=single,
	breaklines=true,
}
\lstdefinestyle{customlatex}{
	basicstyle=\footnotesize\ttfamily,
	%basicstyle=\small\ttfamily,
}
\lstdefinestyle{customc}{
	breaklines=true,
	frame=tb,
	language=C,
	xleftmargin=0pt,
	showstringspaces=false,
	basicstyle=\small\ttfamily,
	keywordstyle=\bfseries\color{green!40!black},
	commentstyle=\itshape\color{purple!40!black},
	identifierstyle=\color{blue},
	stringstyle=\color{orange},
}
\lstdefinestyle{custommatlab}{
	captionpos=t,
	breaklines=true,
	frame=tb,
	xleftmargin=0pt,
	language=matlab,
	showstringspaces=false,
	%basicstyle=\footnotesize\ttfamily,
	basicstyle=\scriptsize\ttfamily,
	keywordstyle=\bfseries\color{green!40!black},
	commentstyle=\itshape\color{purple!40!black},
	identifierstyle=\color{blue},
	stringstyle=\color{orange},
}

%wymiar tekstu (bez żywej paginy)
\textwidth 160mm \textheight 247mm

%ustawienia pakietu pgfplots
\pgfplotsset{
	tick label style={font=\scriptsize},
	label style={font=\small},
	legend style={font=\small},
	title style={font=\small}
}

\def\figurename{Rys.}
\def\tablename{Tab.}

%konfiguracja liczby pływających elementów
\setcounter{topnumber}{0}%2
\setcounter{bottomnumber}{3}%1
\setcounter{totalnumber}{5}%3
\renewcommand{\textfraction}{0.01}%0.2
\renewcommand{\topfraction}{0.95}%0.7
\renewcommand{\bottomfraction}{0.95}%0.3
\renewcommand{\floatpagefraction}{0.35}%0.5

\pagestyle{fancy}
\lhead{{\textsf{Dokumentacja \\ Projektowa}}}
%\lhead{}\textbf
%\rhead{}
\cfoot{\textbf{\textsf{\thepage}}}
\setlength{\parindent}{1cm}
\setcounter{secnumdepth}{4}

\newcommand{\myparagraph}[1]{\paragraph{#1}\mbox{}\\}

\begin{document}
\begin{titlepage}
\begin{center}
\resizebox{!}{1.5cm}{\bf Cloud Factory}\\
\resizebox{!}{1.5cm}{\bf Studio}\\
\end{center}
\vspace{3cm}
\begin{center}
{\bf \resizebox{!}{0.8cm}{Dokumentacja projektowa} \vskip 0.1cm}
\end{center}
\vspace{3cm}
\begin{center}
{\bf \Huge Aplikacja Analizuj�ca Stan Fizyczny U�ytkownika \\ \vskip 0.5cm \resizebox{!}{1.5cm}{Health Together} \vskip 0.1cm}
\end{center}
\vspace{3cm}
\begin{center}
{\large {\bf Wersja 2.0} \\ \Large{\bf Autorzy:} \\ \large Gie�dowski Daniel \\ \large Kuc Piotr \\ \large R�a�ski Antoni \\ \large �wierczek Arkadiusz  \par}
\end{center}
\vspace*{\stretch{6}}
\begin{center}
\bf{\large{Warszawa, 28 Maja 2017\vskip 0.1cm}}
\end{center}
\end{titlepage}

\tableofcontents 
\vspace{3cm}
\begin{table}[h!]
	\centering
	\begin{tabular}{|c|c|l|}
		\hline
		Data Modyfikacji & Numer Wersji & Opis Modyfikacji \\ \hline
		09.05.2017 & 0.1 & Utworzenie dokumentacji \\ \hline
		12.05.2017 & 0.2 & Wst�p i zakres projektu \\ \hline
		13.05.2017 & 0.3 & Organizacja projektu \\ \hline
		15.05.2017 & 0.4 & Procedury sterowania i kontroli \\ \hline
		16.05.2017 & 1.0 & Dokumentacja wst�pna \\ \hline
		20.05.2017 & 1.1 & S�ownik poj��, \\
		 & & Poprawki: Organizacja projektu \\ \hline
		 26.05.2017 & 1.2 & Analiza ryzyka \\ \hline
		 28.05.2017 & 1.3 & Plan i harmonogram projektu \\ \hline
		 31.05.2017 & 2.0 & Wersja ostateczna \\ \hline
	\end{tabular}
\end{table}
\newpage
\section{Wst�p}

Celem dokumentu jest przedstawienie zarz�dzania realizacj� projektu aplikaji "Health Together" dla firmy Capple. Firma Capple zajmuje si� dystrybucj� wysokiej jako�ci aplikacji na platform� Apple Watch. Wraz z rozrostem firmy niemo�liwym sta�s si� samodzielna realizacja wszystkich projekt�w powsta�o zapotrzebowanie na zlecanie ich rozwijania podwykonawcom.

Z powod�w opisanych powy�ej podj�ta zosta�a decyzja o realizacji aplikacji "Health Together" dla firmy Capple przez firm� Cloud Factory Studio. Aplikacja zrealizowana zostanie zgodnie z wszystkimi wymaganiami biznesowymi. B�dzie ona wygodna zar�wno dla klient�w, umo�liwiaj�c dost�p do rzetelnych danych o ich stanie fizycznym, jak i pracownik�w pracuj�cych w help-desku, udost�pniaj�c im stosowny interfejs.

Aplikacja spe�nia� b�dzie wszystkie wymagania biznesowe zwi�zane z prawem odno�nie prywatno�ci.
\newpage
\section{Zakres projektu}
\subsection{Aspekt Biznesowy}

Dzi�ki wprowadzeniu na rynek aplikacji "Health Together" dla firmy Capple mo�liwe b�dzie osi�gni�cie poni�szych mo�liwo�ci biznesowych:
\begin{itemize}
\item Zwi�kszenie popularno�ci firmy, dzi�ki wprowadzeniu nowatorskiej aplikacji.
\item Powi�kszenie przewagi rynkowej nad konkurencj�, dzi�ki  zape�nieniu dot�d niezaspokojonej niszy rynkowej.
\end{itemize}
\subsection{Aspekt U�ytkowy}
Dla ko�cowego u�ytkownika aplikacji aplikacja daje nast�puj�ce korzy�ci:
\begin{itemize}
\item Otrzymywanie rzetelnych informacji na temat swojego aktualnego stanu fizycznego.
\item Du�a wygoda korzystania i mobilno�� umo�liwiaj� kontrol� zdrowia w ka�dej chwili.
\item Wykrywanie stan�w zagra�aj�cych zdrowiu i �yciu oraz alarmowanie odpowiednich s�u�b.
\item Dost�p do najnowszych algorytm�w obliczania i aproksymowania wska�nik�w zdrowia pacjenta.
\end{itemize}
\subsection{Aspekt Techniczny}
System oparty b�dzie na nast�puj�cym modelu:
\begin{itemize}
\item Aplikacja stanowi�ca interfejs wizualny na platform� Apple Watch korzystaj�ca z udost�pnionych przez urz�dzenie sposob�w zbierania danych biometrycznych zaprogramowana za pomoc� j�zyka Swift z wykorzystaniem sieci neuronowych do analizy danych.
\item Zewn�trzna baza danych zawieraj�ca zebrane przez aplikacj� informacje.
\item Komunikacja z baz� i alarmowanie zrealizowane za pomoc� komunikacji urz�dzenia z sieci� internetow� (za po�rednictwem Wi-fi) oraz z sieci� kom�rkow� (za po�rednictwem karty SIM).
\end{itemize}
\newpage
\section{Organizacja projektu}
\subsection{Komitet steruj�cy}
Celem komitetu steruj�cego jest czuwanie nad kierunkiem rozwoju projektu oraz podejmowanie najwa�niejszych decyzji biznesowych, kt�re maj� bezpo�redni wp�yw na produkt. Podczas ka�dego etapu projektu, osoby nale��ce do komitetu powinny nadzorowa� post�py i dotychczasowy progres. Komitet powo�uje oraz bezpo�rednio nadzoruje kierownika projektu.
\begin{table}[h!]
	\centering
	\begin{tabular}{|l|l|l|}
		\hline
		\bf Stanowisko & \bf Pracownik & \bf Obowi�zki \\ \hline
		Przewodnicz�cy & Adam Michulski & \tabitem Zwo�ywanie zebra� komitetu \\ 
		 & & \tabitem Podejmowanie kluczowych decyzji \\ 
		 & & steruj�cych  w sytuacjach kryzysowych \\ \hline
		Dyrektor techniczny & Zbigniew Gor�czka & \tabitem Koordynacja architektury systemu i \\
		 & & jego jako�ci \\
		 & & \tabitem Kontrola zgodno�ci projektu ze \\
		 & & wszelkimi normami oraz certyfikatami \\
		 & & bezpiecze�stwa \\
		 & & \tabitem Podejmowanie innych kluczowych \\
		 & & decyzji technicznych \\ \hline
		Konsultant techniczny & Przedstawiciel firmy Capple & \tabitem Wykonywanie niezale�nego audytu i \\
		 & & ekspertyz projektu realizowanego \\
		 & & systemu \\
		 & & \tabitem Wyja�nianie kwestii spornych i \\
		 & & u�yteczno�ci odpowiednich technologii \\ \hline
	\end{tabular}
\end{table}
\subsection{Zespo�y projektowe}
\subsubsection{Lider zespo�u}
Ka�dy zesp� sk�ada si� z maksymalnie 6 pracownik�w. Ka�dy zesp� posiada lidera, kt�ry reprezentuje zesp� przed kierownikiem projektu. Wszelka komunikacja mi�dzy zespo�ami odbywa si� za pomoc� lider�w. Lider powinien by� uznanym autorytetem w�r�d wszystkich cz�onk�w zespo�u oraz powinien posiada� najwi�ksze do�wiadczenie.
\subsubsection{Cz�onek zespo�u}
Od cz�onk�w poszczeg�lnych zespo��w wymaga si� dobrej znajomo�ci dziedziny, w kt�rej si� specjalizuje. Z za�o�enia cz�onek zespo�u musi pracowa� w ma�y zespole i raportowa� post�p przed liderem zespo�u.
\subsubsection{Zesp� analityk�w}
Przygotowanie analizy proces�w, wymaga�, koszt�w i interfejs�w u�ytkownika. Do obowi�zk�w cz�onk�w tego zespo�u b�dzie nale�a�o tworzenie dokumentacji biznesowej, wspomaganie programist�w oraz przygotowanie plan�w testu aplikacji.
\subsubsection{Zesp� developer�w}
Zesp� ten wykorzystuj�c sprawdzone technologie zaprojektuje oraz wykona aplikacj� na zegarek. Zak�ada si�, �e dost�pna b�dzie gotowa dokumentacja dost�pnych czujnik�w przygotowana przez tw�rc�w inteligentnego zegarka. Od deweloper�w oczekuje si� doskona�ej znajomo�ci technologii chmurowych oraz podstawowej wiedzy w zakresie sztucznych sieci neuronowych. Przed cz�onkami zosta�o postawione zadanie przygotowania dodatkowo serwera zbieraj�cego oraz analizuj�cego dost�pne dane.
\subsubsection{Zesp� tester�w}
Zesp� tester�w zostanie podzielony na dwa mniejsze podzespo�y. Jeden b�dzie odpowiedzialny za bezpo�rednie testowanie modu��w zgodnie z wcze�niej przygotowan� dokumentacj�,a drugi za testy end2end. Pracownicy b�d� mieli za zadanie wychwytywa� i zg�asza� zaistnia�e problemy oraz b��dy w dzia�aniu aplikacji bezpo�rednio do zespo�u developer�w.
\subsubsection{Zesp� administrator�w}
Projekt zak�ada wymian� danych mi�dzy u�ytkownikami, a serwerem. W tym celu nale�y powo�a� zesp� administrator�w zaznajomiony z zaawansowanymi zasadami bezpiecze�stwa. W gr� b�d� wchodz� dane osobowe klient�w, kt�rych strata mo�e narazi� firmy na wielopoziomowe straty. Do obowi�zk�w administrator�w b�dzie nale�a�o zarz�dzanie serwerami z danymi oraz dbanie o bezpiecze�stwo informacji.
\subsection{Kierownik projektu}
Kierownikiem projektu zostaje osoba wyznaczona przez komitet steruj�cy. Do jego obowi�zku nale�y bezpo�rednie kontrolowanie projektu. Od kierownika wymaga si� podejmowania trafnych decyzji kierowniczych, dobr� znajomo�� mo�liwo�ci u�ywanych technologii oraz posiadania umiej�tno�ci zarz�dzania zasobami ludzkimi.
 \begin{table}[h!]
 	\centering
 	\begin{tabular}{|l|l|}
 		\hline
 		\bf Stanowisko & \bf Pracownik \\ \hline
 		Kierownik projektu & Marysia Wajs \\ \hline
 		Lider zespo�u analityk�w & El�bieta Staropolska \\ \hline
 		Lider zespo�u developer�w & Micha� Jaworski \\ \hline
 		Lider zespo�u tester�w & Miko�aj Galopski \\ \hline
 		Lider zespo�u administrator�w & Pawe� Kupski \\ \hline
 	\end{tabular}
\end{table}
\newpage
\section{Procedury sterowania i kontroli projektu}
\subsection{Procedury komunikacji}
\subsubsection{Procedury komunikacji wewn�trz firmy}

\myparagraph{Komunikacja elektroniczna}

Komunikacja wewn�trz firmy odbywa si� za po�rednictwem platformy Slack - narz�dzia do wygodnej wsp�pracy zespo�owej. Ka�dy zesp� i dzia� ma sw�j dedykowany kana�, pozwalaj�cy na niezak��con� komunikacj� i szybk� reakcj� na zaistania�e problemy. Opr�cz tego istnieje szereg og�lnych kana��w firmowych, dzi�ki kt�rym ka�dy upowa�niony do takich informacji pracownik mo�e zorientowa� si�, na jakim etapie znajduj� si� prac� innego zespo�u. Konto ka�dego pracownika zostaje stworzone i zweryfikowane przez cz�onka zespo�u administrator�w przed do��czeniem danego u�ytkownika do kana�u, a wszelkie deklaracje oraz decyzje wyra�ane w tym kanale komunikacji s� traktowane jako jednoznaczne stanowisko nadawcy wiadomo�ci. R�wnie� administratorzy odpowiadaj� za przydzielanie uprawnie� do do��czania do istniej�cych kana��w firmowych.

Platforma Slack udost�pnia swoje aplikacje na wiele platform, co umo�liwia komunikacj� tak�e w przypadku, gdy danego pracownika nie ma w danym momencie w biurze.

\myparagraph{Komunikacja interpersonalna}
\label{comm_wew}

W ka�dym momencie dnia, je�li zaistanieje taka potrzeba, kierownik zespo�u mo�e zarz�dzi� indywidualne spotkanie z wybranym pracownikiem w celu rozwi�zania zauwa�onych przez niego problem�w.

Raz w tygodniu, lider ka�dego zespo�u organizuje spotkanie dla swoich bezpo�rednich podw�adnych w salach konferencyjnych. Termin takiego spotkania jest ustalany wewn�trz zespo�u. Spotkania te maj� na celu raportowanie o aktualnym stanie prac, planowanie pracy w najbli�szych dniach a tak�e rozwi�zywanie problem�w na poziomie zespo�owym.

Prace na najbli�szy okres s� przydzielane w wyniku dyskusji wewn�trz zespo�u, w kt�rej decyduj�cy g�os ma kierownik zespo�u.

Liderzy zespo��w spotykaj� raz w tygodniu, w terminie wst�pnie ustalonym na czwartek, godz. 16.00, z Kierownikiem Projektu. Raportowane s� na tym spotkaniu post�py prac ka�dego z zespo��w oraz ca�ego projektu. Analizowane s� ryzyka i nowe wymagania projektodawcy, przeprowadzana jest r�wnie� analiza obecnego stanu zasob�w i ich planowanego zu�ycia w najbli�szej przysz�o�ci, w wyniku kt�rej ustalana jest strategia na najbli�szy tydzie�. Spotkania te maj� charakter oficjalny i z ka�dego spotkania powstaje kr�tki dokument streszczaj�cy jego przebieg i wypunktowuj�cy powzi�te decyzje.

Przewodnicz�cy Komitetu Steruj�cego zarz�dza spotkanie Komitetu z Kierownikiem Projektu oraz Pe�nomocnikiem do spraw jako�ci nie rzadziej ni� raz na 6 tygodni. Na spotkaniu Komitet Steruj�cy otrzymuje informacje o aktualnym stanie prac, zu�yciu zasob�w, napotkanych problemach i planowanych dzia�aniach w najbli�szej przysz�o�ci. Pe�nomocnik do spraw jako�ci przedstawia dokumenty raportujace kolejne kontrole jako�ci. Je�eli zu�ycie zasob�w, jako�� wytworzonego produktu lub stan prac nie pokrywa si� z tym przewidzianym w Harmonogramie Prac, Komitet Steruj�cy mo�e powzi�� wszelkie decyzje, jakie uwa�a za s�uszne w celu poprawy sytuacji.

\subsubsection{Procedury komunikacji z projektodawc�}
\myparagraph{Komunikacja elektroniczna}

Cz�onkowie zespo�u wsparcia (helpdesku) odpowiedzialni s� za kontakt z klientem; ka�e z nich posiada sw�j osobisty adres e-mail w domenie firmowej.  Mail ten ma przyk�adow� posta� \textit{akowalczyk@healthtogether.com.pl}. Opr�cz tego ka�dy z tej grupy pracownik�w ma sw�j s�u�bowy telefon. Poprzez te dwa kana�y komunikacji klient mo�e skontaktowa� si� z firm� w dowolnym momencie podczas godzin pracy. Firma uzgadnia z klientem maksymalny czas na odpowied�, wynosz� w przypadku wiadomo�ci e-mail 5 dni roboczych. Ka�dy nast�pny dzie� zw�oki, je�li omawiana sprawa dotyczy krytycznych kwestii dla projektu, oznacza� b�dzie op�nienie prac nad projektem nie z winy nadawcy wiadomo�ci i strona opiesza�a w udzielaniu odpowiedzi b�dzie musia�a liczy� si� z poniesiem koszt�w tego op�nienia uwzgl�dnionymi w zawieranej umowie.
\myparagraph{Komunikacja interpesonalna}

W sk�ad Komitetu Steruj�cego wchodzi tak�e przedstawiciel zleceniodawcy. Jest on obecny na spotkanaich tej grupy i uczestniczy w podejmowaniu taktycznych dezycji odno�nie projektu.
Opr�cz tego, ustanawiane zostan� poczas akceptacji planu prac terminy spotka� Kierownika Projektu z kontrolerami jako�ci klienta. Spotkania te b�d� odbywa� si� nie cz�ciej ni� raz na miesi�c, a ich celem jest zaprezentowanie na �ywo aktualnych prac nad aplikacj� i dostosowanie przysz�ych zmian do sugestii klienta.
\myparagraph{Zg�aszanie postulat�w i b��d�w}

W trakcie trwania projektu, opr�cz bezpo�redniej komunikacji, klient mo�e zg�asza� nowe postulaty dotycz�ce niekt�rych funkcji. W p�niejszej fazie projektu, gdy zostanie udost�pniona klientowi wersja alfa aplikacji, b�dzie m�g� on te� zg�asza� znalezione b��dy. 

Aby zapewni� profesjonaln� i wygodn� dla obu stron obs�ug� b��d�w i postulat�w, zostanie uruchomiony firmowy \textit{Mantis } - system zg�aszania i �ledzenia b��d�w (ang. \textit{bug tracker}). Za po�rednictwem tej platformy u�ytkownik b�dzie m�g� zg�asza� postulaty, �ledzi� post�py aktualnie otworzonych zg�osze� jak i ich histori� czy powi�zania mi�dzy nimi. Pracownicy zespo�u wparcia, odpowiedzialni za opi�k� nad Mantisem, b�d� mi�dzy innymi akceptowa� zg�oszenia, przypisywa� odpowiednim zespo�om jak i je zamyka�, gdy zostan� rozwi�zane.

\subsubsection{Procedury komunikacji z u�ytkownikiem aplikacji}
Gdy aplikacja wejdzie w faz� beta, zostanie udost�pniona u�ytkownikom - posiadaczom smartwatcha \textit{Apple Watch}. Od tego momentu, do zada� zespo�u deweloper�w b�dzie nale�a�o utrzymanie aplikacji, a w tym tak�e rozwi�zywanie problem�w zg�oszonych przez u�ytkownik�w dotycz�cych dzia�ania aplikacji na konkretnych wersjach systemu. Aplikacja b�dzie dost�pna przez aplikacj� \textit{Capple Store}, za pomoc� kt�rej u�ytkonwnicy ko�cowi b�d� mogli zar�wno pobra� najnowsz� wersj� aplikacji, jak i j� ocenia� i zg�asza� poprawki. Do obowi�zk�w zespo�u wsparcia (helpdesku) zalicza� b�dzie si� r�wnie� komunikacja z posiadaczami smartwacha przez platform� \textit{Apple Watch}.

\subsection{Procedury zapewnienia i kontroli jako�ci}
\subsubsection{Ustalenie oczekiwanych norm jako�ci}

Przed rozpocz�ciem prac przygotowywany jest niniejszy Plan Realizacji projektu, kt�rego zawarto�� jest uzgadniana z projektodawc�. Do��czany do tego dokumentu jest Harmonogram Prac z terminami i oczekiwanymi kosztami kolejnych etap�w Projektu. Oddzielnie tworzony jest r�wnie� dokument definiuj�cy odpowiedzialno�ci i kary podmiot�w: zlejac�cego wykonanie (zleceniobiorcy, firmy Capple) i wykonuj�cego Projekt (Cloud Factory Studio). Wszystkie dokumenty musz� zosta� podpisane przez upowa�nionych do takiej czynno�ci przedstawicieli firmy Capple oraz Cloud Factory Studio przed pocz�tkiem prac i s� punktem odniesienia w razie wyst�pienia spornych kwestii podczas realizacji Projektu.

\subsubsection{Zapewnienie jako�ci}
Aby zapewni� jak najlepsz� jako�� wytwarzanego oprogramowania, zostan� zaimplementowane nast�puj�ce elementy �rodowiska pracy:
\myparagraph{Stacje robocze}

Ka�de ze stanowisk b�dzie wyposa�one w stacj� robocz� umo�liwiaj�c� komfortow� prac�. W parz� z wysok� jako�ci� podzespo��w komputerowych b�dzie sz�a r�wnie� wygoda u�ytkowania zapewniona przez minimum dwa monitory o przek�tnej ekranu 27'' dla ka�dego pracownika oraz indywidualnie wybierany fotel z oferty na stronie \textit{foteleDlaBiur.com}. 

Aby zapewni� bezawaryjno�� stacji roboczych, ich u�ytkownicy nie b�d� mieli mo�liwo�ci samodzielnej instalacji oprogramownia. Je�li kt�rych z pracownik�w mia�by personalne preferencje co do program�w, kt�rych chce lub potrzebuje u�ywa�, musi zg�osi� to administratorowi przez odpowiedni kana� Slacka, lub, je�li dotyczy to wi�kszej cz�ci firmy, przez prywatny firmowy dzia� Mantisa.

Firma zapenwnia swoim pracownikom wykupione licencje na powszechnie u�ywane aplikacje, takie jak Microsoft 365, zestaw narz�dzi deweloperskich firmy JetBrains czy WinRar.
\myparagraph{�rodowisko pracy}

Poszczeg�lne zespo�y b�d� pracowa�y w niedalekiej odleg�o�ci od siebie, przy czym w �adnym pomieszczeniu nie b�dzie wi�cej jak 4 stanowiska pracy. Pomieszczenia s� klimatyzowane, przestronne i maj� dost�p do wielu �r�de� �wiat�a. Ka�dy dzia� w firmie posiada osobn� kuchni� i �azienk�. Kuchnia umo�liwia przechowywanie i podgrzewanie posi�k�w, a firma zapewnia codziennie napoje i �wie�e owoce. Opr�cz tego, do wsp�lnego u�ytku wewn�trzfirmowego zostanie udost�pniony pok�j do odpoczynku - z miejscem do rozm�w oraz gier na konsoli.

\myparagraph{Infrastruktura}

Aby maksymalizowa� produktywno��, firma musi mie� ci�g�y dost�p do sieci oraz baz danych. Bezawaryjno�� systemu jest zapewniona przez codzienne kontrole (godz 19.30) infrastruktury firmy przez zesp� administrator�w  oraz ich regularne spotkania (raz na tydzie�, termin ustalany przez lidera zespo�u administrator�w) w celu zapewnienia przejrzystej i wydajnej architektury sieci i serwer�w.

\myparagraph{Kompetencje pracownik�w}

Wszyscy pracownicy wyra�aj�cy tak� ch��, mog� wzi�� udzia� w bezp�atnym szkoleniu z u�ywania platformy Slack, Mantis oraz u�ywanego wewn�trz firmy oprogramowania CRM, aby porusza� si� swobodnie i efektywnie w tych �rodowiskach. Opr�cz tego, co miesi�c, w poniedzia�ki, b�d� odbywa�y si� szkolenia z oprogramowania smartwatchy dla programist�w i sztucznych sieci neuronowych oraz z kursy rozwijaj�ce umiej�tno�ci mi�kkie dla os�b odpowiedzialnych za kontakt z klientem. 

\subsubsection{Kontrola jako�ci}
Zarz�d firmy powo�uje Pe�nomocnika do Spraw Jako�ci. Jest on odpowiedzialny za kontrol� jako�ci wytwarzanego projektu informatycznego. Tworzy dokument - Plan jako�ci, w kt�rym okre�la oczekiwane jako�ciowe rezultaty pracy firmy oraz spos�b nadzoru nad nimi. Pe�nomocnika do Spraw Jako�ci ma wgl�d do wszelkich dokument�w oraz kodu oraz uczestniczy w spotkaniach Kierownika Projektu z Zarz�dem firmy, na kt�rych to spotkaniach prezentuje dokument b�d�cy efektem kontroli nast�puj�cych kwestii:

\begin{enumerate}
	\item Jako�� kodu;
	\item Jako�� dokumentacji;
	\item Zgodno�� post�p�w prac z Harmonogramem Prac;
	\item Zgodno�� wykorzystania zasob�w z planem;
	\item Przestrzeganie standard�w;
	\item Wynik audyt�w;
	\item Ocen� realizacji poliyki jako�ciowej firmy
\end{enumerate} 

\myparagraph{Kontrola jako�ci kodu}

Zesp� deweloper�w ma obowi�zek tworzenia kodu wysokiej jako�ci, natomiast zadaniem zespo�u tester�w jest stworzenie takiego zestawu test�w, kt�ry pokryje wszystkie mo�liwe przypadki. B�d� tworzone testy:

\begin{enumerate}
	\item Testy jednostkowe;
	\item Testy integracyjne;
	\item Testy systemowe;
	\item Testy	wydajno�ciowe.

\end{enumerate} 

Pe�nomocnik do Spraw Jako�ci ocenia kod obu tych grup. Uwzgl�dnia w swoim raporcie nast�puj�ce czynniki:

\begin{enumerate}
	\item Czytelno�� kodu;
	\item Podzia� kodu na modu�y i pakiety;
	\item Pokrycie test�w jednostkowych.
\end{enumerate} 
Aby umo�liwi� �atw� wsp�prac� programist�w oraz kontrol� kodu, utworzone zotaje repozytorium GIT pod adresem https://github.com/CloudFactoryStudio/HealthTogether. 

Opr�cz tego firma korzysta z narz�dzia ci�g�ej integracji (Continuous Integration): Jenkins, kt�ry w na bie��co informuje o stanie kodu w repozytorium.


\myparagraph{Kontrola jako�ci dokumentacji}

Deweloperzy maj� obowi�zek tworzenia, obok nowych modu��w, dokumentacji opisuj�cych aspekty techniczne i u�ytkowe utworzonych modu��w. Pe�nomocnik do Spraw Jako�ci sprawdza zgodno�� dokumentacji analitycznej i technicznej ze stanem rzeczywistym, jak i jej transparentno�� i dok�adno��.

\subsection{Kontrola zgodno�ci post�p�w prac oraz wykorzystania bud�etu}
Obowi�zek kontroli zgodno�ci post�p�w prac oraz wykorzystania bud�etu przypada Kierownikowi Projektu. G��wnym narz�dziem, kt�re wykorzystuje podczas kontroli jest wenw�trzny firmowy system CRM. Ka�dy pracownik ma w nim swoje konto (zak�adane przez zeesp� administrator�w) oraz na bie��co raportuje bie��ce prace; zapisywana w bazie danych jest informacja jakie zadanie pracownik wykonywa� danego dnia wraz z ilo�ci� czasu, jakie dane zadanie poch�on�o. Wszelkie dodatkowe wydatki, poniesione przez firm�, takie jak wydatki na infrastruktur� firmy zg�aszane przez administrator�w, r�wnie� s� w tym programie zapisywane. Kierownik Projektu eksportuje z programu raporty miesi�czne i tygodniowe, na podstawie kt�rych ocenia, czy post�py i wydatki mieszcz� si� w ich zaplanowanych wielko�ciach.

W przypadku stwierdzenia nieprawid�owo�ci, Kierownik Projektu, po konsultacji z Komitetem Steruj�cym, mo�e podj�� dezycj� o jednej lub wielu nast�puj�cych czynno�ciach:

\begin{enumerate}
	\item Przesuni�cia personalne w ramach zespo��w
	\item Zmniejszenie zakresu prac
	\item Wprowadzenie nadgodzin i skr�cenie urlop�w
\end{enumerate} 

Do danycyh uzyskanych z systemu CRM dost�p ma r�wnie� zleceniodawca.

\subsection{Kontrola zmian}

Dzi�ki systemowi zg�aszania i �ledzenia b��d�w Mantis klient oraz pracownicy firmy mog� w szybki spos�b reagowa� na zmian� wymaga�. Aby ograniczy� niekorzystny wp�yw zmian w projekcie, nak�ada si� obowi�zek natychmiastowego poinformowania Kierownika Projektu, bezpo�renio lub poprzez platform� Slack o zaistnia�ej sytuacji od razu po jej wykryciu przez pracownik�w zespo�u wsparcie w firmie, kt�rzy zajmuj� si� monitorowaniem systemu Mantis.
\subsubsection{Procedury kontroli zmian}

Wszelkie zmiany musz� by� udokumentowane. Ewidencj� zmian prowadzi Kierownik Projektu, kt�ry na ich podstawie tworzy dokument, kt�ry jest dog��bnie analizowany podczas spotka� Komitetu i wtedy te� jest ustalana odpowiednia reakcja na te zmiany.

Zmiany w projekcie mog� wynika� z:
	
\begin{enumerate}
	\item Zmiany wymaga� funkcjonalnych zleceniodawcy;
	\item Zmiany wymaga� technologicznych zleceniodawcy;
	\item B��d�w w specyfikajchach i dokumentacjach;
	\item Nowych uregulowa� prawnych i gospodarczych;
	\item Nag�ych zmian w dyspozycyjno�ci pracownik�w;
	\item Nag�ych zmian w dost�pno�ci sprz�tu i technologii;
	\item Nag�ych zmian w materialnych i sprz�towych zasobach firmy;
	\item Naruszenia obowi�zuj�cych standard�w.
\end{enumerate} 

Kierownik na podstawie ka�dej zmiany przedstawia Komitetowi dokument zawieraj�cy:

\begin{enumerate}
	\item Dat� zg�oszenia konieczno�ci zmiany;
	\item Przyczyn� konieczno�ci zmiany;
	\item Oczekiwany rezultat zmiany;
	\item Zgrubn� kalkulacj� kosztu wprowadzenia zmiany;
	\item Konsekwencje wprowadzenia zmiany na za�o�one cele;
	\item Plan wprowadzania zmiany.
\end{enumerate} 

Komitet steruj�cy po zapoznianiu si� z dokumentem decyduje o koniecznych krokach kt�re nale�y podj�� w celu minimalizacji koszt�w wprowadzenia zmiany; w szczeg�lno�ci ocenia, czy zmiana jest konieczna, czy spos�b wprowadzenia jej zaproponowany przez Kierownika spe�nia za�o�one wymogi jako�ci oraz bud�etu. Przedstawiciel klienta mo�e zmian� zaakceptowa�, odroczy� lub odrzuci�.

Je�li zmiana zosta�a zaakcpetowana, Komitet wprowadza do planu wprowadzenia zmiany zaproponowanego przez Kierownika ewentualne korekty, po czym Kierownik przyt�puje do realizacji planu. Zostaje przygotowany projest wprowadzenia zmiany, kt�ry streszcza spos�b jej wprowadzenia oraz testowania, razem z modu�ami powi�zanymi. 

\subsection{Rozwi�zywanie problem�w}
W firmie panuj� przejrzyste oraz powszechnie dost�pne dla zainteresowanych zasady pracy. Ka�dy pracownik ma obowi�zek zapozna� si� z nimi i je zaakcpetowa�. Zasady te narzucaj� pewne normy i ich przestrzeganie ma za zadanie pom�c w unikaniu konflikt�w w firmie. Je�li jednak takie wyst�pi�, istniej� procedury ich rozwi�zywania, opisane ju� cz�ciowo w sekcji \ref{comm_wew}. Za rozwi�zywanie konflikt�w na najni�szym szczeblu odpowiedzialni s� liderzy zespo��w; musz� to by� osoby potrafi�ce dostrzec sytuacje konfliktowe i  umiej�tnie je rozwi�za�. W pierwszej kolejno�ci odbywaj� si� indywidualne rozmowy z pracownikiem kt�rego dotyczy dana sytuacj�. Je�li jest to wymagane, problem jest poruszany w gronie zespo�u na cotygodniowych spotkaniach raportuj�cych stan prac. Je�li problem wci�� nie udaje si� rozwi�za�, informowany jest o tym Kierownik Projektu, kt�ry podejmuje ostateczn� decyzj� dotycz�c� istoty konfliktu.

Je�li problem nie jest natury personalnej, lecz sprz�towej lub technicznej, pracownik mo�e to zg�osi� odpowiedniemu zespo�owi poprzez plarform� Mantis do u�ytku wewn�trzfirmowego.
\newpage
\section{Plan projektu}
\subsection{Etapy projektu}
 \begin{table}[h!]
	\centering
	\begin{tabular}{|l|l|l|l|l|}
		\hline
		Numer & Nazwa etapu & Planowana data & Planowana data & Zaanga�owane osoby \\
		etapu & & rozpocz�cia & zako�czenia & \\ \hline 
		1. & Uzgodnienie & 05.06.2017 & 12.06.2017 & \tabitem Komitet steruj�cy \\
		 & wymaga� & & & \tabitem Przedstawiciele klienta \\
		 & & & & \tabitem Kierownik projektu \\
		 & & & & \tabitem Zesp� analityk�w \\ \hline
		 2. & Analiza projektu & 12.06.2017 & 10.07.2017 & \tabitem Kierownik projektu \\
		 & & & & \tabitem Zesp� analityk�w \\
		 & & & & \tabitem Zesp� wsparcia \\ \hline
		 3. & Projektowanie i & 03.07.2017 & 25.09.2017 & \tabitem Dyrektor techniczny \\
		 & implementacja & & & \tabitem Kierownik projektu \\
		 & & & & \tabitem Zesp� developer�w \\
		 & & & & \tabitem Zesp� wsparcia \\ \hline
		 4. & Testowanie & 28.08.2017 & 02.10.2017 & \tabitem Konsultant techniczny \\
		 & & & & \tabitem Kierownik projektu \\
		 & & & & \tabitem Zesp� tester�w \\
		 & & & & \tabitem Zesp� wsparcia \\ \hline
		 5. & Podsumowanie i & 02.10.2017 & 09.10.2017 & \tabitem Komitet steruj�cy \\
		 & odbi�r & & & \tabitem Przedstawiciele klienta \\
		 & & & & \tabitem Kierownik projektu \\
		 & & & & \tabitem Zesp� analityk�w \\ \hline
		 6. & Utrzymanie i & 09.10.2017 & 09.10.2022 & \tabitem Zesp� administrator�w \\
		 & wsparcie & & & \tabitem Zesp� wsparcia \\ \hline
	\end{tabular}
\end{table}
\subsubsection{Uzgodnienie wymaga�}
Celem pierwszego etapu projektu jest precyzyjne okre�lenie wymaga� klienta. W czasie tego etapu bardzo wa�na b�dzie wsp�praca z klientem w celu zebrania jak najwiekszej ilo�ci informacji odno�nie planowanego wygl�du, dzia�ania i specyfikacji aplikacji.
\subsubsection{Analiza projektu}
Podczas tego etapu zesp� analityk�w ma za zadanie stworzy� na podstawie informacji zebranych od klienta wysokiej jako�ci dokumentacj� biznesow�. W razie jakichkolwiek w�tpliwo�ci zesp� powinien skontaktowa� si� z klientem. Po uko�czeniu analizy biznesowej zostanie stworzona tak�e wysokiej jako�ci dokumentacja techniczna.
\subsubsection{Projektowanie i implementacja}
W etapie tym nast�pi stworzenie szczeg�owej dokumentacji technicznej oraz projektu aplikacji, a nast�pnie jej implementacja. Etap zacznie si� jeszcze na tydzie� przed zako�czeniem analizy projektu, kiedy gotowa b�dzie ju� analiza biznesowa i cz�� analizy tecznicznej. 
\subsubsection{Testowanie}
Celem etapu b�dzie gruntowne przetestowanie dzia�ania aplikacji w celu wykrycia wszystkich b��d�w. Etap rozpocznie si� na miesi�c przed zako�czeniem poprzedniego w celu umo�liwienia zespo�owi developer�w poprawy wykrytych nieprawid�owo�ci. Efektem etapu b�dzie szczeg�owy raport. Przeprowadzpne zostan� tak�e testy akceptacyjne w obecno�ci konsultanta technicznego.
\subsubsection{Podsumowanie i odbi�r}
W tym etapie odb�dzie spotkanie podsumowuj�ce projekt oraz odebranie aplikacji przez klienta.
\subsubsection{Utrzymanie i wsparcie}
Ostatnim etapem projektu b�dzie utrzymanie i konserwacja gotowej aplikacji, jej serwer�w i baz danych, a tak�e kontakt z u�ytkownikami w okresie 5 lat od oddania aplikacji.
\subsection{Harmonogram trwania etap�w}
\begin{table}[h!]
	\centering
	\begin{tabular}{|l|l|l|l|l|l|l|l|l|l|l|l|l|l|l|l|l|l|l|l l}
		Etap & \multicolumn{19}{l}{Tygodnie} \\ \hline
		 & \tiny 05. & \tiny 12. & \tiny 19. & \tiny 26. & \tiny 03. & \tiny 10. & \tiny 17. & \tiny 24. & \tiny 31. & \tiny 07. & \tiny 14. & \tiny 21. & \tiny 28. & \tiny 04. & \tiny 11. & \tiny 18. & \tiny 25. & \tiny 02. & \tiny 09. & .. \\
		 & \tiny 06 & \tiny 06 & \tiny 06 & \tiny 06 & \tiny 07 & \tiny 07 & \tiny 07 & \tiny 07 & \tiny 07 & \tiny 08 & \tiny 08 & \tiny 08 & \tiny 08 & \tiny 09 & \tiny 09 & \tiny 09 & \tiny 09 & \tiny 10 & \tiny 10 & .. \\ \hline
		 1. &{\cellcolor{darkgray!25}}&&&&&&&&&&&&&&&&&& \\ \hline
		 2. &&{\cellcolor{darkgray!25}}&{\cellcolor{darkgray!25}}&{\cellcolor{darkgray!25}}&{\cellcolor{darkgray!25}}&&&&&&&&&&&&&& \\ \hline
		 3. &&&&&{\cellcolor{darkgray!25}}&{\cellcolor{darkgray!25}}&{\cellcolor{darkgray!25}}&{\cellcolor{darkgray!25}}&{\cellcolor{darkgray!25}}&{\cellcolor{darkgray!25}}&{\cellcolor{darkgray!25}}&{\cellcolor{darkgray!25}}&{\cellcolor{darkgray!25}}&{\cellcolor{darkgray!25}}&{\cellcolor{darkgray!25}}&{\cellcolor{darkgray!25}}&&& \\ \hline
		 4. &&&&&&&&&&&&&{\cellcolor{darkgray!25}}&{\cellcolor{darkgray!25}}&{\cellcolor{darkgray!25}}&{\cellcolor{darkgray!25}}&{\cellcolor{darkgray!25}}&& \\ \hline
		 5. &&&&&&&&&&&&&&&&&&{\cellcolor{darkgray!25}}& \\ \hline
		 6. &&&&&&&&&&&&&&&&&&&{\cellcolor{darkgray!25}}&{\cellcolor{darkgray!25}} \\ \hline
	\end{tabular}
\end{table}
\subsection{Harmonogram aktywno�ci zespo��w}
\begin{table}[h!]
	\centering
	\begin{tabular}{|l|l|l|l|l|l|l|l|l|l|l|l|l|l|l|l|l|l|l|l l}
		Zesp� & \multicolumn{19}{l}{Tygodnie} \\ \hline
		& \tiny 05. & \tiny 12. & \tiny 19. & \tiny 26. & \tiny 03. & \tiny 10. & \tiny 17. & \tiny 24. & \tiny 31. & \tiny 07. & \tiny 14. & \tiny 21. & \tiny 28. & \tiny 04. & \tiny 11. & \tiny 18. & \tiny 25. & \tiny 02. & \tiny 09. & .. \\
		& \tiny 06 & \tiny 06 & \tiny 06 & \tiny 06 & \tiny 07 & \tiny 07 & \tiny 07 & \tiny 07 & \tiny 07 & \tiny 08 & \tiny 08 & \tiny 08 & \tiny 08 & \tiny 09 & \tiny 09 & \tiny 09 & \tiny 09 & \tiny 10 & \tiny 10 & .. \\ \hline
		\tiny Analityk�w &{\cellcolor{darkgray!25}}&{\cellcolor{darkgray!25}}&{\cellcolor{darkgray!25}}&{\cellcolor{darkgray!25}}&{\cellcolor{darkgray!25}}&&&&&&&&&&&&&{\cellcolor{darkgray!25}}& \\ \hline
		\tiny Developer�w &&&&&{\cellcolor{darkgray!25}}&{\cellcolor{darkgray!25}}&{\cellcolor{darkgray!25}}&{\cellcolor{darkgray!25}}&{\cellcolor{darkgray!25}}&{\cellcolor{darkgray!25}}&{\cellcolor{darkgray!25}}&{\cellcolor{darkgray!25}}&{\cellcolor{darkgray!25}}&{\cellcolor{darkgray!25}}&{\cellcolor{darkgray!25}}&{\cellcolor{darkgray!25}}&&& \\ \hline
		\tiny Tester�w  &&&&&&&&&&&&&{\cellcolor{darkgray!25}}&{\cellcolor{darkgray!25}}&{\cellcolor{darkgray!25}}&{\cellcolor{darkgray!25}}&{\cellcolor{darkgray!25}}&& \\ \hline
		\tiny Wsparcia &&{\cellcolor{darkgray!25}}&{\cellcolor{darkgray!25}}&{\cellcolor{darkgray!25}}&{\cellcolor{darkgray!25}}&{\cellcolor{darkgray!25}}&{\cellcolor{darkgray!25}}&{\cellcolor{darkgray!25}}&{\cellcolor{darkgray!25}}&{\cellcolor{darkgray!25}}&{\cellcolor{darkgray!25}}&{\cellcolor{darkgray!25}}&{\cellcolor{darkgray!25}}&{\cellcolor{darkgray!25}}&{\cellcolor{darkgray!25}}&{\cellcolor{darkgray!25}}&{\cellcolor{darkgray!25}}&&{\cellcolor{darkgray!25}}&{\cellcolor{darkgray!25}} \\ \hline
		\tiny Administrator�w &&&&&&&&&&&&&&&&&&&{\cellcolor{darkgray!25}}&{\cellcolor{darkgray!25}} \\ \hline
	\end{tabular}
\end{table}
\newpage
\section{Analiza ryzyka}
\subsection{Spis znanych ryzyk}

\begin{table}[h!]
	\noindent\rule{16cm}{0.4pt}
	\begin{tabular}[t]{ll}
		\textbf{Zagro�enie nr 1:} & Zbyt optymistyczne planowanie \\
		Mo�liwe skutki: & 
		\begin{tabular}[t]{@{}l@{}}
			\tabitem Niezrealizowanie planu w terminie \\
			\tabitem Zwi�kszenie koszt�w 
		\end{tabular}  \\
		Wp�yw: & Du�y \\
		Prawdopodobie�stwo zaistnienia: & Niskie\\
		
	\end{tabular}
\end{table}



\begin{table}[h!]
	\noindent\rule{16cm}{0.4pt}
	\begin{tabular}[t]{ll}
		\textbf{Zagro�enie nr 2:} & Awaria chmury obliczeniowej\\
		Mo�liwe skutki: & 
		\begin{tabular}[t]{@{}l@{}}
			\tabitem Brak mo�liwo�ci u�ytkowania oprogramowania\\ 
			\tabitem Straty finansowe\\
		\end{tabular}  \\
		Wp�yw: & Ogromny \\
		Prawdopodobie�stwo zaistnienia: & �rednie\\
		
	\end{tabular}
\end{table}

\begin{table}[h!]
	\noindent\rule{16cm}{0.4pt}
	\begin{tabular}[t]{ll}
		\textbf{Zagro�enie nr 3:} & B��dy w oprogramowaniu\\
		Mo�liwe skutki: & 
		\begin{tabular}[t]{@{}l@{}}
			\tabitem Wydanie wadliwego produktu\\ 
			\tabitem Kary finansowe\\
			\tabitem Op�nienia w projekcie\\
		\end{tabular}  \\
		Wp�yw: & Ogromny \\
		Prawdopodobie�stwo zaistnienia: & �rednie\\
		
	\end{tabular}
\end{table}

\begin{table}[h!]
	\noindent\rule{16cm}{0.4pt}
	\begin{tabular}[t]{ll}
		\textbf{Zagro�enie nr 4:} & Braki w zasobach ludzkich\\
		Mo�liwe skutki: & 
		\begin{tabular}[t]{@{}l@{}}
			\tabitem Op�nienia w projekcie\\
			\tabitem Straty finansowe\\
			\tabitem Wydanie wadliwego produktu\\
		\end{tabular}  \\
		Wp�yw: & Du�y \\
		Prawdopodobie�stwo zaistnienia: & Du�e\\
		
	\end{tabular}
\end{table}

\begin{table}[h!]
	\noindent\rule{16cm}{0.4pt}
	\begin{tabular}[t]{ll}
		\textbf{Zagro�enie nr 5:} & Awaria sprz�tu \\
		Mo�liwe skutki: & 
		\begin{tabular}[t]{@{}l@{}}
			\tabitem Op�nienia w projekcie\\
			\tabitem Straty finansowe\\
		\end{tabular}  \\
		Wp�yw: & Du�y \\
		Prawdopodobie�stwo zaistnienia: & Niskie\\
		
	\end{tabular}
\end{table}


\begin{table}[h!]
	\noindent\rule{16cm}{0.4pt}
	\begin{tabular}[t]{ll}
		\textbf{Zagro�enie nr 6:} & Cyberatak\\
		Mo�liwe skutki: & 
		\begin{tabular}[t]{@{}l@{}}
			\tabitem Dost�p do poufnych informacji przez niepowo�ane osoby \\
			\tabitem Kary finansowe \\
			\tabitem Zmiana informacji w bazie danych \\
			\tabitem Wy��czenie chmury obliczeniowej \\
		\end{tabular}  \\
		Wp�yw: & Ogromny \\
		Prawdopodobie�stwo zaistnienia: & Niskie\\
		
	\end{tabular}
\end{table}

\begin{table}[h!]
	\noindent\rule{16cm}{0.4pt}
	\begin{tabular}[t]{ll}
		\textbf{Zagro�enie nr 7:} & Sabota� pracownika\\
		Mo�liwe skutki: & 
		\begin{tabular}[t]{@{}l@{}}
			\tabitem Wydanie wadliwego produktu \\
			\tabitem Wy��czenie chmury obliczeniowej \\
		\end{tabular}  \\
		Wp�yw: & Ogromny \\
		Prawdopodobie�stwo zaistnienia: & Niskie\\
		
	\end{tabular}
\end{table}

\begin{table}[h!]
	\noindent\rule{16cm}{0.4pt}
	\begin{tabular}[t]{ll}
		\textbf{Zagro�enie nr 8:} & Konflikty i nieporozumienie w zespole \\
		Mo�liwe skutki: & 
		\begin{tabular}[t]{@{}l@{}}
			\tabitem Wydanie wadliwego produktu \\
			\tabitem Op�nienia \\
			\tabitem Zmiany organizacyjne w zespo�ach \\
		\end{tabular}  \\
		Wp�yw: & �redni \\
		Prawdopodobie�stwo zaistnienia: & �rednie\\
		
	\end{tabular}
\end{table}

\begin{table}[h!]
	\noindent\rule{16cm}{0.4pt}
	\begin{tabular}[t]{ll}
		\textbf{Zagro�enie nr 9:} & Niedost�pno�� miejsca pracy \\
		Mo�liwe skutki: & 
		\begin{tabular}[t]{@{}l@{}}
			\tabitem Utrudniona komunikacja z innymi cz�onkami projektu \\
		\end{tabular}  \\
		Wp�yw: & Niski \\
		Prawdopodobie�stwo zaistnienia: & Niskie\\
		
	\end{tabular}
\end{table}

\newpage
\subsection{Macierz ryzyk}

\begin{table}[h!]
	
	\begin{tabular}{c|c|c|c|c|c}
		\textbf{Prawdopodobie�stwo:} & & & & & \\ \hline
		Du�e & & &  {\cellcolor{red!25}} &  {\cellcolor{red!25}}4 &  {\cellcolor{red!25}}\\ \hline 
		�rednie & & & 8 & {\cellcolor{red!25}} &  {\cellcolor{red!25}} 3, 2 \\ \hline
		Niskie & & 9 & & 5, 1 &  {\cellcolor{red!25}} 6, 7\\ \hline
		Bardzo niskie & & & & & \\ \hline
		\textbf{Wp�yw: }: & Bardzo niski & Niski & �redni & Du�y & Ogromny \\
		
	\end{tabular}
\end{table}
\newpage
\subsection{Zdefiniowanie prawdopodobnych zagro�e�}

\begin{table}[h!]
	\begin{tabular}[t]{ll}
		\textbf{Zagro�enie nr 2:} & Awaria chmury obliczeniowej\\
		Dzia�ania prewencyjne: & 
		\begin{tabular}[t]{@{}l@{}}
			\tabitem Ograniczenie ilo�ci przetwarzanych danych do minimum \\
			\tabitem Przeniesienie cz�ci oblicze� na inn� chmur� \\
		\end{tabular}  \\
		
	\end{tabular}
\end{table}

\begin{table}[h!]
	\noindent\rule{16cm}{0.4pt}
	\begin{tabular}[t]{ll}
		\textbf{Zagro�enie nr 3:} & B��dy w oprogramowaniu\\
		Dzia�ania prewencyjne: & 
		\begin{tabular}[t]{@{}l@{}}
			\tabitem Wyd�u�ony czas na etap testowania \\
			\tabitem Zatrudnienie wi�kszej ilo�ci tester�w \\
			\tabitem Zwi�kszenie bud�etu przeznaczonego na audyt kodu \\
		\end{tabular}  \\
		
	\end{tabular}
\end{table}

\begin{table}[h!]
	\noindent\rule{16cm}{0.4pt}
	\begin{tabular}[t]{ll}
		\textbf{Zagro�enie nr 4:} & Braki w zasobach ludzkich\\
		Dzia�ania prewencyjne: & 
		\begin{tabular}[t]{@{}l@{}}
			\tabitem Umo�liwienie pracy zdalnej \\
			\tabitem Wdro�enie elastycznych godzin pracy \\
			\tabitem Przyj�cie pewnego prawdopodobie�stwa niedost�pno�ci\\ 
			pracownik�w podczas planowania \\
			\tabitem Okre�lenie dost�pno�ci innych pracownik�w i przydzielenie\\
			ich na czas okre�lony do innego zespo�u
		\end{tabular}  \\
		
	\end{tabular}
\end{table}




\section{S�ownik u�ytych poj��}
\begin{table}[h!]
	\centering
	\begin{tabular}{|l|l|}
		\hline
		\bf Termin & \bf Opis \\ \hline
		Klient & Frima zamawiaj�ca aplikacj�, "Capple" \\ \hline
		Pracownik & Osoba zatrudniona w firmie "Cloud Factory" bior�ca udzia� w projekcie \\ \hline
		Aplikacja & Tworzona aplikacja o nazwie "Health Together" przez firm� "Cloud Factory" \\
		  & na zlecenie firmy "Capple" \\ \hline
	\end{tabular}
\end{table}
\end{document}


